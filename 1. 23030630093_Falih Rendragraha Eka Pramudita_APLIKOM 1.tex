\documentclass[a4paper,10pt]{article}
\usepackage{eumat}

\begin{document}
\begin{eulernotebook}
\eulerheading{Pendahuluan dan Pengenalan Cara Kerja EMT}
\begin{eulercomment}
Selamat datang! Ini adalah pengantar pertama ke Euler Math Toolbox
(disingkat EMT atau Euler). EMT adalah sistem terintegrasi yang
merupakan perpaduan kernel numerik Euler dan program komputer aljabar
Maxima.

- Bagian numerik, GUI, dan komunikasi dengan Maxima telah dikembangkan
oleh R. Grothmann, seorang profesor matematika di Universitas
Eichstätt, Jerman. Banyak algoritma numerik dan pustaka software open
source yang digunakan di dalamnya.

- Maxima adalah program open source yang matang dan sangat kaya untuk
perhitungan simbolik dan aritmatika tak terbatas. Software ini
dikelola oleh sekelompok pengembang di internet.

- Beberapa program lain (LaTeX, Povray, Tiny C Compiler, Python) dapat
digunakan di Euler untuk memungkinkan perhitungan yang lebih cepat
maupun tampilan atau grafik yang lebih baik.

Yang sedang Anda baca (jika dibaca di EMT) ini adalah berkas notebook
di EMT. Notebook aslinya bawaan EMT (dalam bahasa Inggris) dapat
dibuka melalui menu File, kemudian pilih "Open Tutorias and Example",
lalu pilih file "00 First Steps.en". Perhatikan, file notebook EMT
memiliki ekstensi ".en". Melalui notebook ini Anda akan belajar
menggunakan software Euler untuk menyelesaikan berbagai masalah
matematika.
\end{eulercomment}
\begin{eulercomment}
Panduan ini ditulis dengan Euler dalam bentuk notebook Euler, yang
berisi teks (deskriptif), baris-baris perintah, tampilan hasil
perintah (numerik, ekspresi matematika, atau gambar/plot), dan gambar
yang disisipkan dari file gambar.

Untuk menambah jendela EMT, Anda dapat menekan [F11]. EMT akan
menampilkan jendela grafik di layar desktop Anda. Tekan [F11] lagi
untuk kembali ke tata letak favorit Anda. Tata letak disimpan untuk
sesi berikutnya.

Anda juga dapat menggunakan [Ctrl]+[G] untuk menyembunyikan jendela
grafik. Selanjutnya Anda dapat beralih antara grafik dan teks dengan
tombol [TAB].

Seperti yang Anda baca, notebook ini berisi tulisan (teks) berwarna
hijau, yang dapat Anda edit dengan mengklik kanan teks atau tekan menu
Edit -\textgreater{} Edit Comment atau tekan [F5], dan juga baris perintah EMT yang
ditandai dengan "\textgreater{}" dan berwarna merah. Anda dapat menyisipkan baris
perintah baru dengan cara menekan tiga tombol bersamaan:
[Shift]+[Ctrl]+[Enter].

\end{eulercomment}
\eulersubheading{Komentar (Teks Uraian)}
\begin{eulercomment}
Komentar atau teks penjelasan dapat berisi beberapa "markup" dengan
sintaks sebagai berikut.

\end{eulercomment}
\begin{eulerttcomment}
   - * Judul
   - ** Sub-Judul
   - latex: F (x) = \(\backslash\)int_a^x f (t) \(\backslash\), dt
   - mathjax: \(\backslash\)frac\{x^2-1\}\{x-1\} = x + 1
   - maxima: 'integrate(x^3,x) = integrate(x^3,x) + C
   - http://www.euler-math-toolbox.de
   - See: http://www.google.de | Google
   - image: hati.png
   - ---
\end{eulerttcomment}
\begin{eulercomment}

Hasil sintaks-sintaks di atas (tanpa diawali tanda strip) adalah
sebagai berikut.

\begin{eulercomment}
\eulerheading{Judul}
\begin{eulercomment}
\end{eulercomment}
\eulersubheading{Sub-Judul}
\begin{eulercomment}
\end{eulercomment}
\begin{eulerformula}
\[
F(x) = \int_a^x f(t) \, dt
\]
\end{eulerformula}
\begin{eulerformula}
\[
\frac{x^2-1}{x-1} = x + 1
\]
\end{eulerformula}
\begin{eulerformula}
\[
\int {x^3}{\;dx}=C+\frac{x^4}{4}
\]
\end{eulerformula}
\begin{eulercomment}
http://www.euler-math-toolbox.de\\
See: http://www.google.de \textbar{} Google\\
image: hati.png\\
\end{eulercomment}
\eulersubheading{}
\begin{eulercomment}
Gambar diambil dari folder images di tempat file notebook berada dan
tidak dapat dibaca dari Web. Untuk "See:", tautan (URL)web lokal dapat
digunakan.

Paragraf terdiri atas satu baris panjang di editor. Pergantian baris
akan memulai baris baru. Paragraf harus dipisahkan dengan baris
kosong.
\end{eulercomment}
\begin{eulerprompt}
>// baris perintah baru dengan >, komentar (keterangan) diawali dengan //
\end{eulerprompt}
\eulerheading{Baris Perintah}
\begin{eulercomment}
Mari kita tunjukkan cara menggunakan EMT sebagai kalkulator yang
sangat canggih.

EMT berorientasi pada baris perintah. Anda dapat menuliskan satu atau
lebih perintah dalam satu baris perintah. Setiap perintah harus
diakhiri dengan koma atau titik koma.

- Titik koma menyembunyikan output (hasil) dari perintah.\\
- Sebuah koma mencetak hasilnya.\\
- Setelah perintah terakhir, koma diasumsikan secara otomatis (boleh
tidak ditulis).

Dalam contoh berikut, kita mendefinisikan variabel r yang diberi nilai
1,25. Output dari definisi ini adalah nilai variabel. Tetapi karena
tanda titik koma, nilai ini tidak ditampilkan. Pada kedua perintah di
belakangnya, hasil kedua perhitungan tersebut ditampilkan.
\end{eulercomment}
\begin{eulerprompt}
>r = 1.25; pi*r^2, 2*pi*r
\end{eulerprompt}
\begin{euleroutput}
  4.90873852123
  7.85398163397
\end{euleroutput}
\eulersubheading{Latihan untuk Anda}
\begin{eulercomment}
- Sisipkan beberapa baris perintah baru
\end{eulercomment}
\begin{eulerprompt}
>panjang = 4; 
>lebar = 6;
\end{eulerprompt}
\begin{eulerttcomment}
 - Tulis perintah-perintah baru untuk melakukan suatu perhitungan yang
\end{eulerttcomment}
\begin{eulercomment}
Anda inginkan, boleh menggunakan variabel, boleh tanpa variabel.

Misalnya akan dicari luas dan keliling persegi panjang dengan
persamaan berikut :\\
- Luas persegi panjang = panjang * lebar\\
- Keliling persegi panjang = 2(panjang + lebar)
\end{eulercomment}
\begin{eulerprompt}
>luas = panjang*lebar, keliling = 2*(panjang + lebar)
\end{eulerprompt}
\begin{euleroutput}
  24
  20
\end{euleroutput}
\end{eulernotebook}
\end{document}
